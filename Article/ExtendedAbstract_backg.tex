%%%%%%%%%%%%%%%%%%%%%%%%%%%%%%%%%%%%%%%%%%%%%%%%%%%%%%%%%%%%%%%%%%%%%%
%     File: ExtendedAbstract_backg.tex                               %
%     Tex Master: ExtendedAbstract.tex                               %
%                                                                    %
%     Author: Andre Calado Marta                                     %
%     Last modified : 27 Dez 2011                                    %
%%%%%%%%%%%%%%%%%%%%%%%%%%%%%%%%%%%%%%%%%%%%%%%%%%%%%%%%%%%%%%%%%%%%%%
% A Theory section should extend, not repeat, the background to the
% article already dealt with in the Introduction and lay the
% foundation for further work.
%%%%%%%%%%%%%%%%%%%%%%%%%%%%%%%%%%%%%%%%%%%%%%%%%%%%%%%%%%%%%%%%%%%%%%

\section{Flight Dynamics}
\label{sec:backg}

\subsection{Frames of Reference}

The first step before describing the dynamics of a commercial aircraft will be to define the frames of reference used to do so. The first frame of reference, on which 4D trajectories are described, corresponds to the WGS84 frame of reference. A second frame of reference corresponding to the aircraft body frame will be used to provide its fast rotational dynamics. Lastly all aerodynamic forces will be applied in the axial directions of the wind frame. This frame is aligned to the wind speed vector relative to the airplane, given by both the angle of attack $\alpha$ and the sideslip angle $\beta$. For these last two frames of reference, a rotation matrix can be defined from the wind frame to the body frame by


\begin{equation}
R_{BW}=
\begin{bmatrix}
c_\alpha c_\beta & -c_\alpha s_\beta & -s_\alpha \\
s_\beta & c_\beta & 0 \\
s_\alpha c_\beta & -s_\alpha s_\beta & c_\alpha
\end{bmatrix}
\label{eq:wind2body}
\end{equation}

To describe the attitude of the plane Euler, roll, pitch and yaw angles, will also be used, namely $\phi\{-\pi,\pi\}$, $\theta \{-\dfrac{\pi}{2},\dfrac{\pi}{2}\}$, $\psi \{-\pi,\pi\}$. From these angles the rotation matrix from the body to the earth frame is given by

\begin{equation}
R_{EB}=
\begin{bmatrix}
c_\theta c_\psi & s_\phi s_\theta c_\psi - c_\phi s_\psi & c_\phi s_\theta c_\psi + s_\phi s_\psi \\
c_\theta c_\psi & s_\phi s_\theta s_\psi + c_\phi c_\psi & c_\phi s_\theta s_\psi - s_\phi c_\psi \\
-s_\theta & s_\phi c_\theta & c_\phi c_\theta
\end{bmatrix}
\label{eq:body2earth}
\end{equation}

\subsection{Fast Dynamics}

\subsection{Translation Dynamics}

\subsection{Actuator Dynamics}
