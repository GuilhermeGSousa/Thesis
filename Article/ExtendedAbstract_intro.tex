%%%%%%%%%%%%%%%%%%%%%%%%%%%%%%%%%%%%%%%%%%%%%%%%%%%%%%%%%%%%%%%%%%%%%%
%     File: ExtendedAbstract_intro.tex                               %
%     Tex Master: ExtendedAbstract.tex                               %
%                                                                    %
%     Author: Andre Calado Marta                                     %
%     Last modified : 27 Dez 2011                                    %
%%%%%%%%%%%%%%%%%%%%%%%%%%%%%%%%%%%%%%%%%%%%%%%%%%%%%%%%%%%%%%%%%%%%%%
% State the objectives of the work and provide an adequate background,
% avoiding a detailed literature survey or a summary of the results.
%%%%%%%%%%%%%%%%%%%%%%%%%%%%%%%%%%%%%%%%%%%%%%%%%%%%%%%%%%%%%%%%%%%%%%

\section{Introduction}
\label{sec:intro}

As the aeronautic industry grows, so is bound to also grow the air traffic dramatically. To answer this issue ATM researchers have proposed over the last few years Trajectory-Based Operations (TBO), a concept allowing the use of 4D trajectories to manage both safety and air capacity. In both the US and Europe, initiatives to put such systems in place are currently being developed and implemented, namely the NextGen by the FAA and SESAR EUROCONTROL. Therefore, in order to adopt this air traffic management paradigm, automation will play a crucial role in 4D guidance control, allowing an aircraft to follow flight plans more accurately.

In order to fully automatize a commercial aircraft to follow a 4D trajectory in cruise conditions, this work will focus on designing and implementing an autopilot capable of controlling the aircraft attitude, improving flight quality and stability in hazardous piloting situations, to be integrated in a Fly-by-Wire system. The ultimate aim of this project will be to focus on auto pilot to provide 4D trajectory guidance to a commercial aircraft. To do so a model based controller is used, unlike in the currently implemented framework of robust control composed of several PID layers. This model based controller distinguishes fast and slow dynamics, using a nonlinear inversion of the fast dynamics to determine the necessary deflections of the control surfaces. 

This method, however, also has some limitations, the main one being that the feedback linearisation requires an exact knowledge of the system model, to obtain an exact inversion of the system. This is not usually feasible, and errors in the model of the airplane will inevitably lead to inversion errors, especially in cases of heavy external disturbances. A solution for this limitation will be proposed, studied and implemented in this work. 

Over the recent years, research in intelligent and adaptive flight control systems has seen a consistent increase, in an attempt to solve these limitations, in order to develop flight systems able to adapt to external disturbances as per \cite{SotA_IFCS}. Of the existing intelligent control techniques used to solve the dependency of model-based control systems on an accurate mathematical model and the uncertainties caused by external disturbances or component failures, neural networks have been the most successful in doing so. Applied to UAV control, research works such as \cite{online_adaptiveNN}, \cite{UAV_adaptive}, \cite{UAV_adaptive2} and \cite{YANG+LIN_Adaptive_Flight_Control} have showed neural networks can be used to increase flight control stability and render flight systems adaptable to disturbances. For this work an online neural network was used to improve a model based flight controller of a commercial aircraft. 

This paper is organized as follows, Section 2  provides the mathematical model used as well as actuator dynamics. Section 3 focuses on the model-based approach used to control the aircraft, as well as a description on the neural network used to improve said control law. This section will also cover the implementation of the NN described previously and the guidance law used to ensure 4D trajectory following. Finally Section 4 shows simulation results of the control approach, and conclusions are given in Section 5.