%%%%%%%%%%%%%%%%%%%%%%%%%%%%%%%%%%%%%%%%%%%%%%%%%%%%%%%%%%%%%%%%%%%%%%
%     File: ExtendedAbstract_abstr.tex                               %
%     Tex Master: ExtendedAbstract.tex                               %
%                                                                    %
%     Author: Andre Calado Marta                                     %
%     Last modified : 2 Dez 2011                                     %
%%%%%%%%%%%%%%%%%%%%%%%%%%%%%%%%%%%%%%%%%%%%%%%%%%%%%%%%%%%%%%%%%%%%%%
% The abstract of should have less than 500 words.
% The keywords should be typed here (three to five keywords).
%%%%%%%%%%%%%%%%%%%%%%%%%%%%%%%%%%%%%%%%%%%%%%%%%%%%%%%%%%%%%%%%%%%%%%

%%
%% Abstract
%%
\begin{abstract}

Following the current ATM industry paradigm shift to allow for commercial aircraft to follow 4D trajectory, in order to increase both safety and air capacity, a controller was design and implemented in this work to reach such a goal in cruise conditions. The control law used is based on feedback linearisation, and a neural network is also implemented in order to restrain errors caused by modelling uncertainties caused by poor estimation of the aircraft parameters, external disturbances or fault systems. The neural network is trained online using the back-propagation algorithm to optimise the nonlinear inversion, resulting in an overall adaptive model-based controller. Simulation results show this controller is able to maintain stability and controllability in conditions that would otherwise render the aircraft unstable.
\\
%%
%% Keywords (max 5)
%%
\noindent{{\bf Keywords:}} non-linear control, feedback linearisation, neural network, flight control, back-propagation \\

\end{abstract}

