%%%%%%%%%%%%%%%%%%%%%%%%%%%%%%%%%%%%%%%%%%%%%%%%%%%%%%%%%%%%%%%%%%%%%%%%
%                                                                      %
%     File: Thesis_Abstract.tex                                        %
%     Tex Master: Thesis.tex                                           %
%                                                                      %
%     Author: Andre C. Marta                                           %
%     Last modified :  2 Jul 2015                                      %
%                                                                      %
%%%%%%%%%%%%%%%%%%%%%%%%%%%%%%%%%%%%%%%%%%%%%%%%%%%%%%%%%%%%%%%%%%%%%%%%

\section*{Abstract}

% Add entry in the table of contents as section
\addcontentsline{toc}{section}{Abstract}

Due to their inherent non-linear dynamics, designing control systems for both rotary and fixed wing aircraft is a non-trivial task, where using linear control approaches often reveal to be inaccurate. Feedback linearization is an approach to non-linear control design that algebraically transforms a non-linear system onto a linear one. From the linearised system linear control techniques can be used. However this approach, as well as other state-of-the art control systems such as model predictive control, backstepping and gain scheduling require the \textit{a priori} exact mathematical model of the system to be controlled \cite{SotA_IFCS}. The lack of such a model leads to errors in the calculation of the model inversion, necessary to implement feedback linearization. Indeed, while feedback linearization control shows good tracking, it has poor disturbance rejection, that ultimately lead to these inversion errors \cite{SotA_ControlAlgorithm}. One solution to this problem that has gained momentum over the last years is the use augmentation algorithms, namely neural networks and other intelligent algorithms, to minimize and cancel this error\cite{NLI+NN}, \cite{NLI+NN_IFCS}, \cite{NLI+NN_chinese}. This thesis therefore aims to discuss and implement the existing alternatives to augment the precision, robustness and overall performance of feedback linearization control.

\vfill

\textbf{\Large Keywords:} non-linear control, feedback linearization, neural network, flight control

