%%%%%%%%%%%%%%%%%%%%%%%%%%%%%%%%%%%%%%%%%%%%%%%%%%%%%%%%%%%%%%%%%%%%%%%%
%                                                                      %
%     File: Thesis_Abstract.tex                                        %
%     Tex Master: Thesis.tex                                           %
%                                                                      %
%     Author: Andre C. Marta                                           %
%     Last modified :  2 Jul 2015                                      %
%                                                                      %
%%%%%%%%%%%%%%%%%%%%%%%%%%%%%%%%%%%%%%%%%%%%%%%%%%%%%%%%%%%%%%%%%%%%%%%%

\section*{Abstract}

% Add entry in the table of contents as section
\addcontentsline{toc}{section}{Abstract}

Following the current ATM industry paradigm shift to allow for commercial aircraft to follow 4D trajectory, in order to increase both safety and air capacity, a controller was designed and implemented in this work to reach such a goal in cruise conditions. The control law used is based on feedback linearisation, and a neural network is also implemented in order to restrain errors caused by modelling uncertainties due to poor estimation of the aircraft parameters, external disturbances or fault systems. The neural network is trained online using the back-propagation algorithm to optimise the nonlinear inversion, resulting in an overall adaptive model-based controller. Simulation results show this controller is able to maintain stability and controllability in conditions that would otherwise render the aircraft unstable.

\vfill

\textbf{\Large Keywords:} non-linear control, feedback linearisation, neural network, flight control, back-propagation

