%%%%%%%%%%%%%%%%%%%%%%%%%%%%%%%%%%%%%%%%%%%%%%%%%%%%%%%%%%%%%%%%%%%%%%%%
%                                                                      %
%     File: Thesis_Background.tex                                      %
%     Tex Master: Thesis.tex                                           %
%                                                                      %
%     Author: Guilherme Sousa                                          %
%                                                                      %
%                                                                      %
%%%%%%%%%%%%%%%%%%%%%%%%%%%%%%%%%%%%%%%%%%%%%%%%%%%%%%%%%%%%%%%%%%%%%%%%

\chapter{Background}
\label{chapter:background}

On this chapter the aircraft model used for this work will be described firstly. A theoretical model of the plane's dynamics will be discussed, and implementation detail will be provided on the chapter \ref{chapter:implementation}. The control strategy used will then follow, giving an overview of the feedback linearisation approach, as well as its limitations, namely its sensitivity to inversion errors and external interferences. An attempt to solve these limitation will be made, suggesting some solutions and finally describing the chosen methodology for this case.



%%%%%%%%%%%%%%%%%%%%%%%%%%%%%%%%%%%%%%%%%%%%%%%%%%%%%%%%%%%%%%%%%%%%%%%%
\section{Airplane Model}
\label{section:background/model}

The work made in this thesis was built on top of the work done by H. Escamilla Nuñez and  F. Mora Camino on 4D trajectory tracking \cite{hector}. The model used in this work of a six degree of freedom transport aircraft will be described in this section. 

\subsection{Frames of reference}
\label{section:background/model/for}
The first step before describing the dynamics of a commercial aircraft will be to define the frames of reference used to do so. The first frame of reference, on which 4D trajectories are described, corresponds to the WGS84 frame of reference. A second frame of reference corresponding to the aircraft's body frame will be used to provide its fast rotational dynamics. Lastly all aerodynamic forces will be applied in the axial directions of the wind frame. This frame is aligned to the wind speed vector relative to the airplane, given by both the angle of attack $\alpha$ and the sideslip angle $\beta$. For these last two frames of reference, a rotation matrix can be defined from the wind frame to the body frame by


\begin{equation}
R_{WB}=
\begin{bmatrix}
c_\alpha c_\beta & -c_\alpha s_\beta & -s_\alpha \\
s_\beta & c_\beta & 0 \\
s_\alpha c_\beta & -s_\alpha s_\beta & c_\alpha
\end{bmatrix}
\label{eq:wind2body}
\end{equation}

To describe the attitude of the plane Euler angles will also be used, namely $\phi\{-\pi,\pi\}; \theta \{-\dfrac{\pi}{2},\dfrac{\pi}{2}\}; \psi \{-\pi,\pi\}$. From these angles the rotation matrix from the body to the earth's frame is given by

\begin{equation}
R_{WB}=
\begin{bmatrix}
c_\theta c_\psi & s_\phi s_\theta c_\psi - c_\phi s_\psi & c_\phi s_\theta c_\psi + s_\phi s_\psi \\
c_\theta c_\psi & s_\phi s_\theta s_\psi + c_\phi c_\psi & c_\phi s_\theta s_\psi - s_\phi c_\psi \\
-s_\theta & s_\phi c_\theta & c_\phi c_\theta
\end{bmatrix}
\label{eq:body2earth}
\end{equation}

\subsection{Fast Dynamics}
\label{section:background/model/fast_dynamics}

The considered actuators of the aircraft are given by $\delta = [\delta_{ail} \delta_{ele} \delta_{rud}]^T$, each applying a torque along an axis of the body frame. These torques are giver by

\begin{equation}
\begin{bmatrix}
L'\\
M\\
N
\end{bmatrix}
= \dfrac{1}{2}\rho S V_a^2\left(
\begin{bmatrix}
bC_l\\
\bar{c}C_m\\
bC_n
\end{bmatrix}
+ C_\delta \delta\right)
\label{eq:torque}
\end{equation}

where $\bar{c}$ and $b$ represent the wing chord length and span respectively, $C_\delta$ and the moment coefficients $[C_l C_m C_n]^T$ are given by

\begin{equation}
C_\delta = 
\begin{bmatrix}
bC_{l\delta_{ail}} & 0 & bC_{l\delta_{rud}} \\
0 & \bar{c}C_{m\delta_{ele}} & 0 \\
bC_{n\delta_{ail}} & 0 & bC_{n\delta_{rud}}\\
\end{bmatrix}
\label{eq:cdelta}
\end{equation}
\begin{equation}
\begin{bmatrix}
C_l\\
C_m\\
C_n
\end{bmatrix} 
=
\begin{bmatrix}
C_{l\beta} \beta + C_{l_p} p \dfrac{b}{2V_a} + C_{l_r} r \dfrac{b}{2V_a}\\
C_{m_0} + C_{m_\alpha} \alpha + C_{m_q} q \dfrac{\bar{c}}{2V_a}\\
C_{n\beta} \beta + C_{n_p} p \dfrac{b}{2V_a} + C_{n_r} r \dfrac{b}{2V_a}
\end{bmatrix}
\label{eq:cmoment}
\end{equation}
Where $p, q, r$ are the body's angular rates ($\Omega = [p q r]^T$). The method of obtaining of the coefficients of equation \ref{eq:cmoment} will be provided in the chapter to follow. Having defined the torques applied to the aircraft the rotational dynamics equation can now be stated as
\begin{gather}
ola\\
ola
\label{eq:fast_dynamics}
\end{gather}

\section{Feedback linearisation}
\label{section:background/NLI}
% Aqui vamos nos!

Feedback linearisation, also known as dynamic inversion, is an approach based on the idea of algebraically transforming a non-linear system into a linear one, from which linear control laws can be used to control the resulting system. Unlike Jacobian linearisation, that assumes linearity of the system around an equilibrium value, feedback linearisation implements a feedback loop that cancels non-linearities of the given system. 



\section{Limitations of feedback linearisation}
\label{section:background/limitations}

\subsection{Control improvement}
\label{section:background/improvements}

\section{Neural Networks}
\label{section:background/NN}



