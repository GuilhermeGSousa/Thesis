%%%%%%%%%%%%%%%%%%%%%%%%%%%%%%%%%%%%%%%%%%%%%%%%%%%%%%%%%%%%%%%%%%%%%%%%
%                                                                      %
%     File: Thesis_Introduction.tex                                    %
%     Tex Master: Thesis.tex                                           %
%                                                                      %
%     Author: Andre C. Marta                                           %
%     Last modified :  2 Jul 2015                                      %
%                                                                      %
%%%%%%%%%%%%%%%%%%%%%%%%%%%%%%%%%%%%%%%%%%%%%%%%%%%%%%%%%%%%%%%%%%%%%%%%

\chapter{Introduction}
\label{chapter:introduction}


Due to their inherent non-linear dynamics, designing control systems for both rotary and fixed wing aircraft is a non-trivial task, where using linear control approaches reveal to be inaccurate. Feedback linearisation is an approach to non-linear control design that algebraically transforms a non-linear system into a linear one. For the linearised system linear control techniques can be used. Note that in this work other equivalent terms to feedback linearisation will be used, namely nonlinear inversion (NLI) and dynamic inversion. 
However this approach, as well as other state-of-the art control approaches such as model predictive control, backstepping and gain scheduling require the \textit{a priori} exact mathematical model of the system to be controlled \cite{SotA_IFCS}. The lack of such an exact model leads to errors in the calculation of the model inversion, necessary to implement feedback linearisation. Indeed, while feedback linearisation control shows good tracking, it is sensitive to parameter uncertainties, that ultimately lead to these inversion errors \cite{SotA_ControlAlgorithm}, and in extreme cases to instability. One solution to this problem that has gained momentum over the last years is the use augmentation algorithms, namely neural networks and other intelligent algorithms, to minimize and cancel the resulting guidance error\cite{NLI+NN}, \cite{NLI+NN_IFCS}, \cite{NLI+NN_chinese}. 

This work aims to discuss the existing alternatives to augment the precision, robustness and overall performance of feedback linearisation control. A second goal of this work will be to extend the solutions used to increase the robustness of feedback linearisation control to provide an adaptive control in cases of system failures and strong external disturbances.



%%%%%%%%%%%%%%%%%%%%%%%%%%%%%%%%%%%%%%%%%%%%%%%%%%%%%%%%%%%%%%%%%%%%%%%%
\section{Motivation}
\label{section:motivation}

As the aeronautic industry grows, so is bound to also grow the air traffic dramatically. To answer this issue, ATM researchers have proposed over the last few years Trajectory-Based Operations (TBO), a concept allowing the use of 4D trajectories to manage both safety and air capacity. In both the US and Europe, initiatives to put such systems in place are currently being developed and implemented, namely the NextGen by the FAA and SESAR EUROCONTROL. Therefore, in order to adopt this air traffic management paradigm, automation will play a crucial role in 4D guidance control, allowing an aircraft to follow flight plans more accurately.

%%%%%%%%%%%%%%%%%%%%%%%%%%%%%%%%%%%%%%%%%%%%%%%%%%%%%%%%%%%%%%%%%%%%%%%%
\section{Topic Overview}
\label{section:overview}

In order to fully automatize a commercial aircraft to follow a 4D trajectory in cruise conditions, this work will focus on designing and implementing an autopilot capable of controlling the aircraft attitude, improving flight quality and stability in hazardous piloting situations, to be integrated in a Fly-by-Wire system. The ultimate aim of this project will be to focus on auto pilot to provide 4D trajectory guidance to a commercial aircraft. To do so a model based controller is used, unlike in the currently implemented framework of robust control composed of several PID layers. This model based controller distinguishes fast and slow dynamics, using a nonlinear inversion of the fast dynamics to determine the necessary deflections of the control surfaces. Unlike PIDs and controllers that use linearisation through the Jacobian using Taylor series expansions of the nonlinear system, that may yield unsatisfactory performance and robustness at a larger range of operating points, a nonlinear inversion where the exact original model is known will also yield a linear model that is the exact representation of the nonlinear aircraft model over a wide range of operating points. 

This method, however, also has some limitations, the main one being that the feedback linearisation requires an exact knowledge of the system model, to obtain an exact inversion of the system. This is not usually feasible, and errors in the model of the airplane will inevitably lead to inversion errors, especially in cases of heavy external disturbances. A solution for this limitation will be proposed, studied and implemented in this work. By using an online neural network, these inversion errors will be corrected, leading to an improved performance and robustness. It will also add an adaptive controller to the aircraft, as it will be able to react and adapt to different types of disturbances.

%%%%%%%%%%%%%%%%%%%%%%%%%%%%%%%%%%%%%%%%%%%%%%%%%%%%%%%%%%%%%%%%%%%%%%%%
\section{Objectives}
\label{section:objectives}

This work can be divided in four progressive goals, that must be sequentially achieved. Firstly, the model of a commercial aircraft will be studied thoroughly, defining both its fast and slow dynamics, as well as methods to obtain its aerodynamic coefficients. Once a model is established, it will be then implemented in a simulation environment, in this case Matlab/Simulink. The first goal will be achieved once the simulation of a commercial aircraft is validated and its results are theoretically coherent. A controller using feedback linearisation will then be designed and tested to control the aircraft plant. The desired objective in this case is to obtain a stable reference following with reduced error.

The third objective of this work will indeed be built on top of the previous two. After studying the different types of errors and external perturbations that the linearised model of the aircraft may be submitted to, its behaviour in such conditions will be discussed. A decrease of the accuracy of the controller and stability are expected of the airplane in these conditions.
These tests will serve not only to verify the impact of a reduced knowledge of the exact model of the plane, but also to test the robustness of the aircraft to external perturbations such as wind and icing conditions. For the fourth and last objective an online neural network will then be used to improve the robustness and stability of the aircraft in cases in which the default NLI controller performed poorly or even became unstable. The overall final objective will be to have an adaptive controller for a commercial aircraft based on feedback linearisation, without the limitations of model based controllers, capable of following 4D trajectories and displaying increased robustness to external disturbances. 






%%%%%%%%%%%%%%%%%%%%%%%%%%%%%%%%%%%%%%%%%%%%%%%%%%%%%%%%%%%%%%%%%%%%%%%%
\section{Thesis Outline}
\label{section:outline}

In chapter \ref{chapter:background}, a theoretical overview of the work will be given. This chapter will introduce the reader to the different tools used in this work, the two most relevant ones being the feedback linearisation method and neural networks. A description of the feedback linearisation technique and theory will then be provided, as well as its limitations. Finally the theory behind the online neural network will be studied and how it can reduce the inversion errors of a feedback linearisation controller.
The following chapter \ref{chapter:implementation} will focus on the implementation of the concepts described previously. Starting with the implementation of the plane model, this section will describe its fast, translation and actuator dynamics, and precise some details in developing a simulation for a commercial aircraft, stating some of its aerodynamic parameters and methods used to obtain them. The controller architecture will then be described, including the inversion algorithm, linear controller and guidance control law. Finally, from both the plane model and its controller comes the neural network architecture used, along with its training algorithm and how it is included to the baseline controller. In chapter \ref{chapter:results} the nonlinear inversion controller is validated and tested in various environments, including disturbances that degrade its performance and robustness. In the same conditions the designed online neural network is included to compare its performance to that of a controller without neural network corrections, finally concluding this study in chapter \ref{chapter:conclusions}.

