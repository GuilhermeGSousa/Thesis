%%%%%%%%%%%%%%%%%%%%%%%%%%%%%%%%%%%%%%%%%%%%%%%%%%%%%%%%%%%%%%%%%%%%%%%%
%                                                                      %
%     File: Thesis_Results.tex                                         %
%     Tex Master: Thesis.tex                                           %
%                                                                      %
%     Author: Andre C. Marta                                           %
%     Last modified :  2 Jul 2015                                      %
%                                                                      %
%%%%%%%%%%%%%%%%%%%%%%%%%%%%%%%%%%%%%%%%%%%%%%%%%%%%%%%%%%%%%%%%%%%%%%%%

\chapter{Results}
\label{chapter:results}

This chapter will be presenting the results in the behaviour of the aircraft for the different solutions proposed to control and stabilise it. Once the goal for the results of this chapter is properly established, the first step will be to validate the model that was implemented. This will be done by controlling the aircraft into cruise conditions using the baseline feedback linearisation error. The effects of disturbances and inversion errors will then be studied regarding their effects in aircraft dynamics. The next step will be to achieve the goal of this thesis and demonstrate the effects of an on-line neural network in reducing the tracking error in the presence of these disturbances. Finally, from the adaptive controller, including the neural network, the guiding law described in chapter \ref{chapter:implementation} in \ref{eq:guidance_law} will be added to follow a given trajectory.


%%%%%%%%%%%%%%%%%%%%%%%%%%%%%%%%%%%%%%%%%%%%%%%%%%%%%%%%%%%%%%%%%%%%%%%%
\section{Model Validation}
\label{section:results/validation}

The goal in this section will be to validate the behaviour of the model in cruise conditions. Assuming cruise conditions comes that

\begin{gather}
	T=D\\
	W=L\\
	L'=M=N=0
\label{eq:cruise_cond}
\end{gather}
In order to verify the model described so far, the required thrust to have cruise conditions will be computed for a given plausible value of $\alpha$. From this point the airspeed of the aircraft can also be computed. The graph of $C_L$ versus alpha was also obtained from its respective neural network, given by \ref{fig:cl_alpha}
\begin{figure}[!htb]
  \centering
  \includegraphics[width=0.75\textwidth]{Figures/CL_alpha.png}
  \caption[$C_L$ versus $\alpha$ graph]{$C_L$ versus $\alpha$ graph from Neural Network}
  \label{fig:cl_alpha}
\end{figure}
From the cruise conditions \ref{eq:cruise_cond}, knowing that $L=\dfrac{1}{2}\rho S V^2 C_L$, solving for the airspeed V comes that

\begin{equation}
V=\sqrt{\dfrac{2mg}{\rho S C_L}}
\label{eq:cruise_speed}
\end{equation}
The required thrust can also be computed from both \ref{eq:cruise_cond}, \ref{eq:cruise_speed} and \ref{eq:cd_cl}, knowing the $C_L$ for a given angle of attack. Proposing some values of $\alpha$, the following results are obtained

\begin{table}[htbp]
  \centering
  \caption{Required cruise conditions for different values of $\alpha$}
    \begin{tabular}{ccccc}
    \toprule
    $\alpha (^o)$ & $C_D$ & $C_L$ & $V (ms^{-1})$ & $T (N)$ \\
    \midrule
    0     & 0.017677131 & 0.0387 & 707.4010791 & 224047.3585 \\
    2     & 0.019379779 & 0.1859 & 322.7613368 & 51133.84325 \\
    4     & 0.023345134 & 0.334 & 240.7952785 & 34283.79709 \\
    6     & 0.029604436 & 0.4828 & 200.279994 & 30076.58604 \\
    \bottomrule

    \end{tabular}%
  \label{tab:cruise_cond}%
\end{table}%


%%%%%%%%%%%%%%%%%%%%%%%%%%%%%%%%%%%%%%%%%%%%%%%%%%%%%%%%%%%%%%%%%%%%%%%%
\section{Feedback linearisation controller}
\label{section:results/fl_contro}

Quest for the optimal solution...

\section{Disturbances and Errors}
\label{section:results/disturbances_errors}


\section{Neural Network}
\label{section:results/NN}

\section{Guidance controller}
\label{section:results/guidance_control}


