%%%%%%%%%%%%%%%%%%%%%%%%%%%%%%%%%%%%%%%%%%%%%%%%%%%%%%%%%%%%%%%%%%%%%%%%
%                                                                      %
%     File: Thesis_Resumo.tex                                          %
%     Tex Master: Thesis.tex                                           %
%                                                                      %
%     Author: Andre C. Marta                                           %
%     Last modified :  2 Jul 2015                                      %
%                                                                      %
%%%%%%%%%%%%%%%%%%%%%%%%%%%%%%%%%%%%%%%%%%%%%%%%%%%%%%%%%%%%%%%%%%%%%%%%

\section*{Resumo}

% Add entry in the table of contents as section
\addcontentsline{toc}{section}{Resumo}

Seguindo a mudança de paradigma que acontece atualmente da industria do tráfego aéro, esta tese de mestrado propõe um controlador para aviões comerciais permitindo o seguimento de trajectórias 4D, aumentando assim a segurança e capacidade num dado espaço aéro. A lei de controlo utilizada é baseada no método de linearização por feedback, e uma rede neuronal é também implementada de modo a reduzir os erros causados por incertezas na modelização do sistema, causados por erros de estimação dos parâmetros do avião, perturbações externas ou até falhas de systema. A rede neuronal é treinada online, utilizando o algoritmo de retro-propagação, de modo a otimizar a inversão não-linear, resultando assim num controlador adaptivo. Os resultados de simulações mostram que este controlador é capaz de manter o sistema estável e controlável em condições em que normalmente tal não é possível.

\vfill

\textbf{\Large Palavras-chave:} controlo não linear, linearização por feedback, rede neural, controlo de voo, back-propagation

